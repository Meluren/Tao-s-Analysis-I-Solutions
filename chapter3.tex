\chapter{Set theory}

\section{Fundamentals}
\begin{enumerate}[Ex. 3.1.1.]
    % Exercise 3.1.1.
    \item[Ex. 3.1.1.] Let $A$, $B$, and $C$ be sets. First, $A = A$ for all $A$ since every
    element of $A$ is an element of $A$. Second, if $A = B$ then $A$ and $B$
    contain the exact same elements, and so $B = A$. Finally, if $A = B$ and 
    $B = C$ then every element of $A$ belong to $B$, and every element of $B$
    belongs to $C$. Hence every element of $A$ belongs to $C$, and vice versa.

    \qed

    % Exercise 3.1.2.
    \item[Ex. 3.1.2.] We have $\varnothing \neq \Set{\varnothing}$ since the empty set has
    no elements, but $\Set{\varnothing}$ has one element, namely $\varnothing$.
    These are both distinct from the set $\Set{\Set{\varnothing}}$ since it
    contains the element $\Set{\varnothing}$ which both of the other sets lack.
    All three are distinct from $\Set{\varnothing, \Set{\varnothing}}$ since
    the latter has the element $\Set{\varnothing}$ which the first two sets
    lack, and the element $\varnothing$ which the third set lacks.

    \qed
    
    % Exercise 3.1.3.
    \item[Ex. 3.1.3.] First $\Set{a,b} = \Set{a} \cup \Set{b}$ immedtiately by Axiom 3.4.
    Next, $A \cup A = A \cup \varnothing = \varnothing \cup A = A$ by Axiom 3.4
    again. Finally, commutativity follows from Axiom 3.4 and commutativity of
    the logical OR operation.

    \qed

    % Exercise 3.1.4.
    \item[Ex. 3.1.4.] If $A \subseteq B$ and $B \subseteq A$ then if $x \in A$ then $x \in B$,
    and if $x \in B$ then $x \in A$. This is exactly Definition 3.1.4. For the final
    statement we already know $A\subseteq C$, given $A\not\subsetneq B$ and 
    $B \subsetneq C$. There is some element $x$ in $B$ that is not an element of $A$.
    Since $B\subseteq C$ (note carefully the symbol used!) we have $x\in C$ and so
    $A \subsetneq C$.

    \qed

    % Exercise 3.1.5
    \item[Ex. 3.1.5.] If $A \subseteq B$ then if $x \in A$ also $x \in B$ and so $A \subseteq A\cap B$,
    whence $A = A \cap B$ for $A\cap B\subseteq A$ by definition.

    If $A\cap B = A$ then if $x \in A\cup B$ then either $x \in B$ and $A \cup B \subseteq B$
    or $x \in A = A\cap B$ and so $x \in B$ and again $A\cup B\subseteq B$. So $A\cup B = B$
    since $B \subseteq A\cup B$ by definition.

    Finally, if $A\cup B = B$ we have for any $x \in A$ also $x \in A\cup B = B$ and so
    $A \subseteq B$.

    \qed

    % Exercise 3.1.6.
    \item[Ex. 3.1.6.] 
    \begin{enumerate}[(a)]
        \item By Lemma 3.1.13 and definition of set intersection.
        \item By Exercise 3.1.5.
        \item By part (b) replacing $X$ by $A$ and noting $A\subseteq A$ for all $A$.
        \item By Exercise 3.3.3. as well as commutativity of $AND$ operation combined wit the
        definition of set intersection.
        \item Associativity of unions and intersections follow from associativity of OR and AND,
        respectively, combined with the definitions of set union and set intersection.
        \item We show the first equality. The second equality is similar. If $x\in A\cap(B\cup C)$
        then $x\in A$ and $x\in B\cup C$. If $x\in B$ then $x\in A\cap B$. If $x\in C$ then
        $x\in A\cap C$. In either case, $x\in (A\cap B)\cup(A\cap C)$. Thus,
        $A\cap(B\cup C) \subseteq (A\cap B)\cup(A\cap C)$. Conversely, if
        $x\in (A\cap B)\cup (A\cap C)$ then if $x\in A\cap B$ we have
        $x\in A\cap(B\cup C)$ for $x\in A$ and $x\in B$. If $x\in A\cap C$ then also
        $x\in A\cap(B\cup C)$ for $x\in A$ and $x\in C$. Regardless, $(A\cap B)\cup(A\cap C)
        \subseteq A\cap(B\cup C)$. All in all, $A\cap(B\cup C) = (A\cap B)\cup(A\cap C)$.
        \item Clearly $A\cup (X-A) \subseteq X$ so we only need show $X\subseteq A \cup (X - A)$.
        But this is true, since if $x\in X$ then either $x\in A$ or $x\not\in A$. In either case,
        $x\in A\cup (X - A)$.

        Similarly, clearly $\varnothing \subseteq A\cap(X-A)$, and we need only show $A\cap(X-A)$ is
        empty. If $x\in A\cap(X - A)$ then $x\in A$ and $x\not\in A$, which is impossible.

        \item We show the first law. The second one is analogous. If $x\in X - (A\cup B)$ then
        $x\in X$ and $x$ is neither in $A$ nor in $B$. Thus $x\in X-A$ and $x\in X-B$. Thus
        $x\in (X-A)\cap(X-B)$. Conversely, if $x\in (X-A) \cap (X-B)$ we have $x\in X$ regardless,
        but also $x\not\in A$ and $x\not\in B$, and so $x\in X-(A\cup B)$.
    \end{enumerate}

    \qed

    % Exercise 3.1.7.
    \item[Ex. 3.1.7.] If $x\in A\cap B$ then $x\in A$ and so $A\cap B\subseteq B$. Also, $x\in B$ and so
    $A\cap B \subseteq B$.

    If $C\subseteq A$ and $C\subseteq B$ we have that if $x\in C$ then $x\in A$ and $x\in B$ and
    so $x\in A\cap B$. Thus, $C\subseteq A\cap B$. Conversely, if $C\subseteq A\cap B$ then if $x\in C$
    then $x\in A$ and $x\in B$, and so $C\subseteq A$ and $C\subseteq B$.

    The other case is done analogously.

    \qed

    % Exercise 3.1.8.
    \item[Ex. 3.1.8.] By the distributive property, $A\cap(A\cup B) = (A\cap A) \cup (A\cap B) = A\cup (A\cap B)$.
    Clearly $A\subseteq A\cup (A\cap B)$. But also, if $x\in A\cup (A\cap B)$ either $x\in A$ and we
    are done, or $x\in A\cap B$ and so $x\in A$ and we are done, for in both cases $x\in A$ and so
    $A\cup (A\cap B) \subseteq A$. All in all, $A = A\cup (A\cap B)$.

    \qed

    % Exercise 3.1.9.
    \item[Ex. 3.1.9.] Suppose $x \in A$. Then $x\not\in B$ and so $x\in X$ since $X = A\cup B$. Thus
    $A\subseteq X - B$. Conversely, if $x\in X - B$ then $x\in X$ and $x\not\in B$. Thus
    $x\in A$ since $X = A\cup B$. Hence, $X- B\subseteq A$. All in all, $A = X - B$. The
    other case is analogous.

    \qed

    % Exercise 3.1.10.
    \item[Ex. 3.1.10.] The sets $A-B$ and $A\cap B$ are disjoint for the first
    contains no elements from $B$ (or is empty) and the second contains element 
    of $B$ (or is empty). The sets $A -B$ and $B -A$ are disjoint for the first 
    contains elements of $A$ (or is empty) and the second contains no elements from
    $A$ (or is empty). The sets $A\cap B$ and $B - A$ are disjoint for the second contains elements
    from $A$ (or is empty) and the first contains no elements from $A$ (or is empty).

    Now, we have $(A - B) \cup (A\cap B) \cup (B - A) = A\cup B$ since any element of the LHS
    is an element of exactly one of the three sets, and in either case it must belong to either
    $A$, $B$, or both. Similarly any element of $A\cup B$ belong either to $A$, $B$, or both, and
    is thus an element of exactly one of the three sets on the LHS.

    \qed

    % Exercise 3.1.11.
    \item[Ex. 3.1.11.] Let $A$ be a set, and for each $x\in A$ let $P(x)$ be a property pertaining to $x$.
    Let us form the statement $Q(x,y)$ which means $P(x)$ is true and $x = y$. Then by the
    Axiom of Replacement we replace each $x$ by $y$ (which is equal to $x$) such that $P(x)$
    is true. This is nothing but the Axiom of Specification.

    \qed
\end{enumerate}


\section{Russel's paradox (Optional)}
\begin{enumerate}[Ex. 3.2.1]
    % Exercise 3.2.1.
    \item For each axiom we need only specify predicate $P$. So for Axiom 3.2
    let $P(x)$ be ``$x$ is a dog and $x$ is not a dog.". h(z) = For Axiom 3.3 let
    $P$ be $P(x) = y$ for the singleton set $\Set{y}$. For Axiom 3.4 let
    $P(x)$ be $x\in A$ or $x \in B$. Axiom 3.5 is already selected based on a
    predicate, we need only add $P(x)$ to be $x\in A$ and $x$ satisfies whatever
    predicate $Q$ that Axiom 3.5 uses. For Axiom 3.6 let $P(x)$ mean just as in
    the previous case, but add again that $x\in A$.

    \qed

    % Exercise 3.2.2.
    \item Let $A$ be a set. If $A\in A$ but $A$ is a set, then we can form
    the singleton set $\Set{A}$. But by the regularity axiom we would require
    that one element of $\Set{A}$ either not be a set, which isn't the case,
    or disjoint from $\Set{A}$, which is not the case.

    \qed

    % Exercise 3.2.3.
    \item The existence of the universal set is given by a vacuous predicate $P$.
    Now, suppose we can form a universal set $\Omega$. Then by the axiom of
    specification the predicate $P$ would then be exactly the predicate described
    in the axiom of universal specification, operating on this universal set $\Omega$. 

    \qed
\end{enumerate}


\section{Functions}
\begin{enumerate}[Ex. 3.3.1.]
    % Exercise 3.3.1.
    \item The definition is reflexive, for $f = f$ for all $f$, since $f$'s domain
    it equal to $f$'s domain, and $f(x) = f(x)$ for all $x\in X$. Symmetry follows
    from symmetry of set equality and equality of objects in the image of $f$.
    If $f = g$ and $g = h$, then $f$, $g$, and $h$ all have the same domain and
    range from transitivity of set equality, and $f(x) = g(x) = h(x)$ from
    transitivity of the objects in the image of $f$ (or $g$ or $h$).

    Finally, for substitution, the domain of $g\circ f$ is the same as
    $\tilde{g}\circ\tilde{f}$ by set substitution, and the same is true of the domain.
    Moreover, if $f = \tilde{f}$ and $g = \tilde{g}$ then by substitution in the range
    the function compositions must be equal.

    Note: All of the above relies on equality defined on the range of $f$, $g$,
    $\tilde{f}$, and $\tilde{g}$ being well-defined, i.e. satisfying the substitution
    axiom.

    \qed
    
    % Exercise 3.3.2.
    \item If $x\neq x'$ then $f(x) \neq f(x')$ since $f$ is injective. Thus,
    since $g$ is also injective, $g(f(x)) \neq g(f(x'))$. The case is similar
    for surjection.

    \qed

    % Exercise 3.3.3.
    \item The empty function is vacuously injective. It is surjective (and bijective) 
    if and only if its range is specified as the empty set.

    \qed

    % Exercise 3.3.4.
    \item Suppose $g\circ f = g\circ\tilde{f}$. The functions $f$ and $\tilde{f}$
    have the same domain and range.  Since $g$ is injective, 
    $g\circ f(x) = g\circ\tilde{f}(x)$ implies $f(x) = \tilde{f}(x)$ for all $x\in X$.
    It does not hold if we drop the assumption that $g$ is injective, for then
    $f$ and $\tilde{f}$ may map some $x$ to different elements which $g$ in turn
    maps to the same element.

    The other case is analogous, with the requirement that $f$ being surjective
    being mandatory as well.

    \qed

    % Exercise 3.3.5.
    \item If $f$ is not injective, $f(x) = f(x')$ for some $x\neq x'$. But then
    $g\circ f(x) = g\circ f(x')$ as well, so it is not injective. It is not
    necessarily true that $g$ also is injective.

    If $g$ is not surjective then immediately $g\circ f$ is not surjective. It
    is not necessarily true that $f$ also is surjective.

    \qed

    % Exercise 3.3.6.
    \item Let $f(x) = y$. Then $f^{-1}(y) = x$ by definition. Thus,
    $f^{-1}(y) = f^{-1}(f(x)) = x$. Similarly $f(f^{-1}(y)) = y$. We conclude
    by letting $h = (f^{-1})^{-1}$ and noting that $f^{-1}\circ h(x) = f^{-1}\circ
    (f^{-1})^{-1}(x) = x$ so that $h = f$. (Here we used the uniqueness of the inverse,
    but this can easily be shown).

    \qed

    % Exercise 3.3.7.
    \item The function $g\circ f$ is bijective from Exercise 3.3.2. We also have
    $g\circ f)^{-1}\circ(g\circ f) = I$ where $I$ is the identity function. Thus,
    by cancelling on the right, we get $(g\circ f)^{-1} = f^{-1}\circ g^{-1}$.

    \qed

    % Exercise 3.3.8.
    \item
    \begin{enumerate}[(a)]
        \item Suppose $X\subseteq Y\subseteq Z$. Then $\imath_{Y\to Z}(\imath_{X\to Y}(x)) =
        \imath_{Y\to Z}(x) = x$. 

        \item The domains and ranges being equal is immediate. Now, let $x\in A$. Then
        $f(\imath_{A\to A}(x)) = f(x) = \imath_{B\to B}(f(x))$.

        \item This is just Exercise 3.3.6.

        \item Define
        \[ h(z) = \begin{cases}
            f(z) & z\in A, \\
            g(z) & z\in B.
        \end{cases} \]

        This is well-defined since $X$ and $Y$ are disjoint. The range of $h$ is
        indeed $Z$ since the range of both $f$ and $g$ is $Z$.
    \end{enumerate}

    \qed
\end{enumerate}


\section{Images and inverse images}
\begin{enumerate}[Ex. 3.4.1.]
    % Exercise 3.4.1.
    \item To avoid confusion let $f^{-1}(V)$ denote the image of $V$ under the
    inverse of $f$, whilst letting $g$ denote the inverse image of $V$ under $f$.

    If $x\in f^{-1}(V)$ then there exists a $y\in V$ such that $f^{-1}(y) = x$,
    or equivalently $y = f(x)$. But then $x\in g(V)$, by definition, and vice versa.

    \qed

    % Exercise 3.4.2.
    \item We have $S \subseteq f^{-1}(f(S))$ since if $x\in S$ then $f(x)\in f(S)$,
    and so $x\in f^{-1}(f(S))$. We also have $f(f^{-1}(U)) \subseteq U$ since if
    $y \in f(f^{-1}(U))$ then there exists an $x \in f^{-1}(U)$ such that $y = f(x)$,
    and so $y \in U$.

    \qed

    % Exercise 3.4.3.
    \item First we show $f(A \cap B) \subseteq f(A) \cap f(B)$. Suppose
    $f(x)\in f(A\cap B)$. Then $x\in A$ and $x\in B$. Thus $f(x)\in f(A)$
    and $f(x)\in f(B)$, so $f(x)\in f(A) \cap f(B)$. This cannot be strengthened
    to an equality. To see why, consider $A = \Set{0}$ and $B=\Set{1}$, and define
    $f : \Set{0,1}\to \Set{0}$ where $f : 0 \mapsto 0$ and $f : 1\mapsto 0$. Then
    $f(A\cap B) = f(\varnothing) = \varnothing \neq \Set{0} = \Set{0} \cap \Set{0}
    = f(A) \cap f(B)$.

    Second we show that $f(A) - f(B) \subseteq f(A - B)$. Suppose $f(x) \in f(A)
    - f(B)$. Then $x\in A$ but $x \not\in B$. Thus $f(x) \in f(A - B)$. This
    cannot be strengthened to eqaulity. Consider the above example again, from
    which we get $\varnothing \neq \Set{0}$.

    Finally, we show that $f(A\cup B) = f(A) \cup f(B)$. Suppose $f(x)\in f(A\cup B)$,
    and let $x \in A$, say. Then $f(x) \in f(A)\subseteq f(A)\cup f(B)$.
    Conversely, suppose $f(x)\in f(A)$, say. Then $x\in A \subseteq A \cup B$ and
    so $f(x)\in f(A\cup B)$.

    \qed

    % Exercise 3.4.4
    \item We show only the first claim; the others are analogous.

    Suppose $x \in f^{-1}(U \cup V)$. Then $f(x) = y$ for some $y \in U \cup V$,
    say $y \in U$. Consequently, $x \in f^{-1}(U)$, by definition, and so
    $x \in f^{-1}(U) \cup f^{-1}(V)$.

    Conversely, if $x \in f^{-1}(U) \cup f^{-1}(V)$, say $x \in f^{-1}(U)$,
    then there exists a $y \in U \subseteq U \cup V$ such that $f(x) = y$.
    Thus, by definition, $x \in f^{-1}(U \cup V)$.

    \qed

    % Exercise 3.4.5.
    \item By Exercise 3.4.2 we need only show $f^{-1}(f(S)) \subseteq S$ if $f$
    is injective, and $S \subseteq f(f^{-1}(S))$ if $f$ is surjective.

    For the first statement, suppose $x \in f^{-1}(f(S))$. Then $f(x) \in f(S)$.
    Suppose now that $x \not\in S$. Since $f$ is injective there exists no
    $x'\in S$ such that $f(x') = f(x)$, but then we would have $f(x) \not\in f(S)$.
    Thus we must have $x \in S$.

    For the second statement, suppose $y \in S$. Then there exists an $x \in
    f^{-1}(S)$ such that $f(x) = y$, since $f$ is surjective. But then
    $y \in f(f^{-1}(S))$.

    \qed

    % Exercise 3.4.6.
    \item Simply define $A = \Set{f^{-1}(\Set{1}) | f \in \Set{0,1}^X}$.
    This is precisely the power set of $X$, as there is a one-to-one
    correspondence between the functions (essentially mapping a truth value
    whether or not an element is contained in the subset) to subsets of $X$.
    For example, $f : x \mapsto 0$ for all $x\in X$ is the empty set and
    $f : x \mapsto 1$ for all $x\in X$ is the entire set $X$.

    \qed

    % Exercise 3.4.7.
    \item Simply define the set $\Set{{Y'}^{X'} | X'\in P(X), Y'\in P(Y)}$,
    where $P(A)$ denotes the power set of $A$. This uses the previous exercise
    and the axiom of specification.

    \qed

    % Exercise 3.4.8.
    \item Let $A$ and $B$ be sets. Form the pair $\Set{A, B}$. Then, by Axiom 3.11,
    $A \cup B = \bigcup\Set{A, B}$.

    \qed

    % Exercise 3.4.9.
    \item If say $y \in \Set{x\in A_\beta | x\in A_\alpha\text{ for all } 
    \alpha\in I}$, then particularly $y \in A_{\beta'}$, and vice versa.

    The equality (3.4) holds for the same reason.

    \qed

    % Exercise 3.4.10.
    \item For the first part suppose $x \in \big(\bigcup_{\alpha\in I} A_\alpha\big)
    \cup \big(\bigcup_{\alpha\in J} A_\alpha\big)$, say $x\in\bigcup_{\alpha\in I} 
    A_\alpha$. Without loss of generality let $x\in A_\beta$ for some particular
    $\beta\in I \subseteq I \cup J$. Then $x \in \bigcup_{\alpha\in I\cup J} A_\alpha$.

    Conversely, if $x \in \bigcup_{\alpha\in I\cup J} A_\alpha$ we have $x\in A_\beta$
    where $\beta \in I \cup J$, say $\beta \in I$. Then $x\in 
    \bigcup_{\alpha\in I} A_\alpha \subseteq \big(\bigcup_{\alpha\in I} 
    A_\alpha\big) \cup \big(\bigcup_{\alpha\in J} A_\alpha\big)$.

    For the second part suppose $I$ and $J$ are non-empty (In order to not violate
    ZF). Then if $x\in \big(\bigcap_{\alpha\in I} A_\alpha\big) \cap 
    \big(\bigcap_{\alpha\in J} A_\alpha\big)$ we have $x \in A_\beta$ for all
    $\beta \in I$ and $x\in A_{\beta'}$ for all $\beta' \in J$. Thus $x\in A_\gamma$
    for all $\gamma$ in $I \cup J$ and so $x\in \bigcap_{\gamma\in I\cup J}
    A_\gamma$.

    Conversely, if $x\in \bigcap_{\alpha\in I\cup J} A_\alpha$ then $x\in A_\gamma$
    for all $\gamma \in I\cup J$, that is, for all $\alpha \in I$ and all 
    $\beta\in J$, we have $x\in A_\alpha$ and $x\in A_\beta$. But this means
    that $x \in \big(\bigcap_{\alpha\in I} A_\alpha\big) \cap 
    \big(\bigcap_{\alpha\in J} A_\alpha\big)$.

    \qed

    % Exercise 3.4.11.
    \item We show only the first claim; the second is analogous.

    If $x \in X - \bigcup_{\alpha \in I} A_\alpha$, then $x \in X$ but
    $x \not\in A_\alpha$ for \emph{any} $\alpha\in I$. Thus $x\in X - A_\alpha$
    for all $\alpha \in I$, or in other words, $x\in\bigcap_{\alpha\in I}
    (X-A_\alpha)$.

    Conversely, if $x \in \bigcap_{\alpha \in I}(X - A_\alpha)$ then
    $x \in X - A_\alpha$ for all $\alpha \in I$ which means $x \not\in A_\alpha$
    for \emph{any} $\alpha \in I$. Thus, $x\not\in\bigcup_{\alpha\in I} A_\alpha$
    and so $x \in X - \bigcup_{\alpha \in I} A_\alpha$.

    \qed
\end{enumerate}


\section{Cartesian products}


\section{Cardinality of sets}
